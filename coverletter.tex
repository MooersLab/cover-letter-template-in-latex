\documentclass[letterpaper,11pt]{letter}
\usepackage{graphicx, fullpage}
\usepackage[dvipsnames]{xcolor}
\usepackage{fancyhdr}
\usepackage{booktabs}
\usepackage[utf8]{inputenc}
%\RequirePackage{booktabs}
%\newenvironment{commontabular}[2]
%  {\begin{tabular}{ll}
%   \end{tabular}}
\pagestyle{fancy}
\fancyhf{}
\chead{  \includegraphics[scale=0.35]{BPletterHead.png}  }
% pdflatex <filename>
%\cfoot{  (\includegraphics [width=\textwidth] {foot.png})  }
 \definecolor{crimson}{RGB}{220,20,60}
\cfoot{  \small \color{crimson}Biomedical Research Center, Room 466, 975 NE 10th St, Oklahoma City, Oklahoma 73104 \\
Office:  (405) 271-8300  Lab:   (405) 271-8313  FAX:  (405) 271-3910 blaine-mooers@ouhsc.edu   \\
https://medicine.ouhsc.edu/academic-departments/biochemistry-and-physiology}
% Some of the article customisations are relevant for this class

% Alternatively, these may be set on an individual basis within each letter environment.
%  end of preamble
\renewcommand{\headrulewidth}{0pt}
\renewcommand{\footrulewidth}{1pt}
%\renewcommand{\footrule}{\color{red}}
\def\mydoubleq#1{``#1''}
\def\mysingleq#1{`#1'}
\headheight 80pt              %% put this outside
\headsep 1pt                 %% put this outside
\voffset  -20pt
\textheight 580pt
\begin{document}
\begin{letter}
\\
\today \\
\\
PLOS Computational Biology\\
\\
RE: manuscript submission \\
\\
Dear Editor,\\

Please find attached the enclosed paper titled \emph{Fifteen quick tips for using coding assistants to configure Emacs} as part of the Education track.
Learning how to use Emacs is on the bucket list of many scientists because of the high-level popularity of org-mode for preparing research articles, books, slides, and posters and for managing projects and time.
Org-mode was developed by an astronomer, Carsten Dominik, early in his career to organize his work.
His use of Emacs has not harmed him, so he had a very successful and is at the top of his field.

This paper provides suggestions on how to utilize on-line coding assistants to help customize Emacs to meet the needs of the user's scientific workflows.
The Emacs initialization file is written in Emacs Lisp which enables rapid development but which also baffles beginners due to its forest of parentheses.
The coding assistants can return code and explain it thereby helping users at all levels overcome many obstacles, thereby saving them time and reducing their frustration.
However, the coding assistants as tutors are limited by their tunnel vision due to a lack of awareness of the full context.
Nonetheless, they are still useful because they never tire of answering prompts.
We focus on on-line coding assistants because their use requires only installing browser plugins.  

We have developed these tips without assistance from chatbots, so the tips are original with us.
Some tips may be relatively obvious to some readers, but others are not so obvious.
However, we did use coding assistants to develop some of the code that we mention in the paper.
We also include links to our software on GitHub and Zenodo that users can utilize to enhance their Emacs learning journey.
We think that this paper will be of broad interest because the tips can be translated to other programming languages and coding assistants.

This paper has not been submitted to any other journal.
We have supplied a list of potential reviewers.

\vskip 1em
{Best regards,} \\
{\includegraphics[width=70mm]{BlainesSignature.png}}\\
{Blaine H. M. Mooers, PhD \\ Associate Professor of Biochemistry and Physiology}

%\cc{} % people this letter is cc-ed to
%\encl{} % list of anything enclosed
%\ps{} % any post scriptums. ``PS'' labels must be put in manually

\end{letter}
\end{document}

